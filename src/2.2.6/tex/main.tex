\documentclass[a4paper, 12pt]{article}

\def \srcdir{tex/}
\def \picdir{pic/}
\def \tbldir{tex/tables/}

\input{\srcdir properties}
\input{\srcdir macros}

\title{
  Лабораторная работа \textnumero \input{\srcdir index}\\
  \textbf{\textquote{\input{\srcdir name}\unskip}}
}
\author{\input{\srcdir author}}
\date{\input{\srcdir date}}

\begin{document}

\maketitle\thispagestyle{fancy}

\subsection*{Цель работы}
Измерить скорости падения шариков при разной температуре жидкости, вычислить\
вязкость жидкости по закону Стокса и расчитать энергию активации.

\subsection*{Оборудование}
\begin{itemize}[noitemsep]
  \item Стеклянный цилиндр с исследуемой жидкостью (глицерин)
  \item Термостат
  \item Секундомер
  \item Горизонтальный компаратор
  \item Микроскоп
  \item Мелкие шарики
\end{itemize}

\section{Теоретическая часть}
\subsection{Энергия активации}
Двойственнный характер свойств жидкостей связан с особенностями дважения их молекул.\
В газах молекулы движутся хаотично, в их расположении отсутствует порядок. В\
кристаллических твёрдых телах частицы колеблются около определённых положений\
равновесия --- узлов кристаллической решётки. В жидкостях, как и в кристаллах, каждая\
молекула находится в потенциальной яме электрического поля. создаваемого окружающими\
молекулами. Молекулы колеблются со средней частотой, близкой к частоте колебаний\
атомов в кристаллических телах ($\sim10^{12}\ Гц$), и с амплитудой, определяемой\
размерами объёма, предоставленного ей соседними молекулами. Глубина потенциальной ямы\
в жидкостях больше средней кинетической энергии колеблюцейся молекулы, поэтому\
молекулы колеблются вокруг более или менее стабильных положений равновесия. Однако\
у жидкостей различие между этими двумя энергиями невeлико, так что молекулы нередко\
выскакивают из \textquote{своей} потенциальной ямы и занимают место в другой. Таким\
образом, молекулы медленно перемещаются внутри жидкости, пребывая часть времени\
около определённых мест равновесия и образуя картину меняющейся со временем\
пространственной решётки, т.е. в жидкости присутствует ближний, но не дальний\
порядок.

Для того чтобы перейти в новое состояние, молекула должна преодолеть участки с\
большой  потенциальной энергией, превышающей среднюю тепловую энергию молекул.\
Для этого тепловая энергия молекул должна --- вследствие флуктуации --- увеличиться\
величину энергии активации $W$/. Температурная зависимость вязкости жидкости
выражается следующей формулой:
\salign{\eta \sim Ae^{\frac{W}{kT}}.}

\subsection{Рассчетная формула вязкости}
На всякое тело, двигающееся в вязкой жидкости, действует сила сопротивления. Для\
ламинарно обтекаемого шарика, движущегося с малой скоростью в безграничной жидкости\
сила сопротивления $F$ определяется формулой
\salign{F = 6\pi\eta rv,}
где $\eta$ --- вязкость жидкости, $v$ --- скорость шарика, $r$ - его радиус.

Рассмотрим свободное падение шарика в вязкой жидкости. На шарик действуют три силы:\
сила тяжести, архимедова сила и сила вязкого трения, зависящая от скорости. По\
второму закону Ньютона:
\salign{V\cg(\rho - \rho_{\text{ж}}) - 6\pi\eta rv = V\rho \frac{\df v}{\df t},}
где $V$ --- объем шарика, $\rho$ --- его плотность, $\rho_{\text{ж}}$ --- плотность жидкости,\
$\cg$ --- ускорение свободного падения. Решая это уравнение, найдём
\salign{v(t) = v_{\text{уст}} - \left[v_{\text{уст}} - v(0)\right]e^{-\frac{t}\tau},}
где $v(0)$ --- скорость шарика в момент начала его дважения в жидкости, $v_{\text{уст}}$\
--- установившаяся скорость, $\tau$ --- время релаксации:
\salign{v_{\text{уст}} = \frac{V\cg(\rho - \rho_{\text{ж}})}{6\pi\eta r} = \frac{2}{36} \cg d^2 \frac{\rho - \rho_{\text{ж}}}{\eta},\quad \tau = \frac{V\rho}{6\pi\eta r} = \frac{2}{36} \frac{d^2\rho}{\eta} = \frac{\rho}{\rho - \rho_{\text{ж}}} \frac{v}{\cg}.}
Как видно из (4), скорость шарика экспоненциально приближается к установившейся\
скорости $v_{\text{уст}}$. Если время падения в несколько раз больше времени релаксации,\
процесс установления скорости можно считать закончившимся.

Измеряя на опыте установившуюся скорость падения шариков $v_{\text{уст}}$ и величины\
$r,\ \rho,\ \rho_{\text{ж}}$, можно определить вязкость жидкости по формуле, следующей из (5):
\salign{\eta = \frac{2}{36} \cg d^2 \frac{\rho - \rho_{\text{ж}}}{v_{\text{уст}}}.}

\begin{figure}[h]
  \begin{center}
    \includegraphics[scale=0.25]{\picdir ustanovka.jpg}
    \caption{Установка}
  \end{center}
\end{figure}

\section{Ход работы}
Результаты измерений и расчета вязкости приведены в \textbf{таб. 1}, результаты\
расчета $Re,\ \tau,\ S$ для оценки справедливости формулы Стокса приведены в \textbf{таб. 2}.\
Формулы для расчета погрешностей отдельных величин: %
\salign{\sigma_v = v\sqrt{\left(\frac{\sigma_t}{t}\right)^2 + \left(\frac{\sigma_l}{l}\right)^2},\quad \sigma_\eta \approx \eta\sqrt{\left(\frac{\sigma_\rho}{\rho - \rho_{\text{ж}}}\right)^2 + 2\left(\frac{\sigma_d}{d}\right)^2 + \left(\frac{\sigma_v}{v}\right)^2},}
\salign{\sigma_{\frac{1}{T}} = \frac{\sigma_T}{T^2},\quad \sigma_{\ln{\eta}} = \frac{\sigma_\eta}{\eta}.}
Наилучшее значение:
\salign{\langle \eta \rangle = \frac{1}{n} \sum_{i=1}^n \eta_i,\quad \sigma_{\text{случ}} = \sqrt{\frac{1}{n(n - 1)} \sum_{i=1}^n (\eta_i - \langle \eta \rangle)^2}.}
Полную погрешность буду считать следующим образом:
\salign{\sigma_{\langle \eta \rangle} = \sqrt{\left(\frac{1}{n}\sum_{i=1}^n \sigma_{\eta_i}\right)^2 + \sigma_{\text{случ}}^2}.}

Результаты расчета величин для построения графика $\ln{\eta}(1/T)$ приведены в \textbf{таб. 3},\
сам график --- на \textbf{рис. 2}. Из (1):
\salign{\ln{\eta} \sim \frac{W}{k} \frac{1}{T} + C,}
где $C$ --- некоторая константа, откуда находим $W$ по коэффициенту наклона графика $\xi$:
\salign{W = \xi k.}
Итоговое значение:
\salign{W = (8,89 \pm 0.18) \cdot 10^{-20}\ Дж,\quad \varepsilon_W = 2\%}

\stbl{1}{Результаты измерений и расчета величины вязкости жидкости}
\stbl{2}{Расчет величин для проверки справедливости формулы Стокса}
\stbl{3}{Величины для графика}
\svg[1]{lab}{График $\ln{\eta}(\frac{1}{T})$}

\subsection*{{Вывод}}
Значения величины $Re$ не превышало допустимого, при котором течение нельзя считать ламинарным.\
Значения $\tau$ и $S$ достаточно малы для расчета скорости, как $v_{\text{уст}}= \frac{l}{t}$. Полученное\
значение W примерно совпадает с табличным $W_0 = 8.7 \cdot 10^{-20}$.
\end{document}